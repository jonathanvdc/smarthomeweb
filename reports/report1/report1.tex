\documentclass[12pt,a4paper]{article}
\usepackage[utf8]{inputenc}
\usepackage{amsmath}
\usepackage{amsfonts}
\usepackage{amssymb}
% \usepackage{all_is_fuck}
\begin{document}
\title{Rapport 1}
\author{Sibert Aerts \\ Ken Bauwens \\ Pieter Hendriks \\ Jonathan Van der Cruysse \\ Mauris Van Hauwe}
\maketitle

\newcommand{\code}[1]{\texttt{#1}}

\section{Status}

We hebben intussen de volgende features ge\"implementeerd:

\begin{itemize}

\item \textbf{REST API}: Jonathan, Mauris \& Sibert

\item \textbf{Database Structuur}: Jonathan, Pieter \& Mauris

\item \textbf{Database Queries}: Jonathan \& Pieter

\item \textbf{Database Integration}: Jonathan, Mauris, Pieter \& Sibert

\item \textbf{Login}: Mauris \& Pieter

\item \textbf{HTML Views}: Sibert, Ken \& Mauris

\end{itemize}

\section{Design}

\subsection{Keuzes}

\subsection{Database schema}

\subsection{UML schema}

Het UML schema zelf is toegevoegd als bijlage. De backend bestaat uit de volgende componenten:

\begin{itemize}

\item \textbf{model classes}:  

\item \textbf{}:

\end{itemize}

\section{Product}

\subsection{Basisvereisten}

\subsection{Extra functionaliteit}

\section{Planning}

\section{Appendix}

\end{document}