\documentclass[12pt,a4paper]{article}
\usepackage[utf8]{inputenc}
\usepackage{amsmath}
\usepackage{amsfonts}
\usepackage{amssymb}
% \usepackage{all_is_fuck}
\begin{document}
\title{Rapport 1}
\author{Sibert Aerts \\ Ken Bauwens \\ Pieter Hendriks \\ Jonathan Van der Cruysse \\ Mauris Van Hauwe}
\maketitle

\newcommand{\code}[1]{\texttt{#1}}

\section{Status}

We hebben intussen de volgende features ge\"implementeerd:

\begin{itemize}

\item Jonathan, Mauris \& Sibert hebben de \textbf{REST API} verzorgd, wat inhoudt dat we ruwe json data uit de database kunnen requesten en hiernaar kunnen posten.

\item Jonathan, Pieter \& Mauris hebben de \textbf{Database Structuur} opgesteld, dit zijn alle huidige tables en constraints hierop.

\item Jonathan \& Pieter hebben de interne \textbf{Database Queries} voorzien, die de server gebruikt om data uit de database te lezen en om data weg te schrijven naar de database.

\item Jonathan, Mauris, Pieter \& Sibert hebben gewerkt aan \textbf{Database Integration}. Dit omvat de meer complexe methodes die data uit de database behandelen.

\item Mauris \& Pieter hebben \textbf{Login} functionaliteit ge\"implementeerd, zodat men op een rudimentaire manier een session kan aanmaken waardoor de gebruiker ge\"identificeerd kan worden door de server. 

\item Sibert, Ken \& Mauris hebben voor rudimentaire \textbf{HTML Views} gezorgd, waardoor men data uit de verschillende tabellen uit de database kan bekijken.

\end{itemize}

\section{Design}

\subsection{Keuzes}

\subsection{Database schema}

\subsection{UML schema}

Het UML schema zelf is toegevoegd als bijlage. De backend bestaat uit de volgende componenten:

\begin{itemize}

\item \textbf{model classes}:  

\item \textbf{}:

\end{itemize}

\section{Product}

\subsection{Basisvereisten}

\subsection{Extra functionaliteit}

\section{Planning}

\section{Appendix}

\end{document}