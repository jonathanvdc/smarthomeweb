\documentclass[12pt,a4paper]{article}
\usepackage[utf8]{inputenc}
\usepackage{amsmath}
\usepackage{amsfonts}
\usepackage{amssymb}
% \usepackage{all_is_fuck}
\begin{document}
\title{Rapport 1}
\author{Sibert Aerts \\ Ken Bauwens \\ Pieter Hendriks \\ Jonathan Van der Cruysse \\ Mauris Van Hauwe}
\maketitle

\newcommand{\code}[1]{\texttt{#1}}

\section{Status}

We hebben intussen de volgende features ge\"implementeerd:

\begin{itemize}

\item REST API

\end{itemize}

\section{Design}

\subsection{Keuzes}

\subsection{Database schema}

\subsection{UML schema}

Het UML schema zelf is toegevoegd als bijlage. De \emph{back-end} bestaat uit de volgende componenten:

\begin{itemize}

\item \textbf{model classes}: het \emph{model} stelt tuples in de database voor. Concreet is het \emph{model} opgesplitst in twee soorten klassen.

\begin{itemize}

\item \textbf{identity classes}: stellen een tuple in de database voor met een unieke \emph{identifier}, die door de \emph{data connection} gegenereerd wordt. Hun data wordt voorgesteld door een instantie van een \emph{data class}. Enkele voorbeelden van \emph{identity classes}: \code{Person}, \code{Sensor} en \code{Location}.

\item \textbf{data classes}: stellen concrete data voor. Deze data komt niet noodzakelijk precies overeen met een tuple in de database, maar altijd wel met een bepaalde deelverzameling daarvan, zonder unieke identifier. Er bestaan twee voorname \emph{use cases} voor deze \emph{data objects}. Enerzijds worden ze gebruikt om, zij het in samenwerking met een \emph{identity object}, zij het onafhankelijk, informatie uit de database voor te stellen. Anderzijds kunnen ze ook gebruikt worden om data van objecten die nog toegevoegd moeten worden aan de database -- en dus nog geen unieke \emph{identifier} hebben -- voor te stellen. \code{PersonData}, \code{LocationData} en \code{Measurement} zijn voorbeelden van \emph{data classes}.

\end{itemize}

\item \textbf{data connection}: spant de brug tussen de database en de \emph{back-end}. De \code{DataConnection} klasse opent en sluit de database, en bedient zich van SQL queries om records aan de database toe te voegen, en deze later weer uit de database te halen. \code{DataConnection} vormt een laag abstractie bovenop directe communicatie met de database: database tuples worden meteen vertaald in \emph{model objects}, en omgekeerd.

\end{itemize}

\section{Product}

\subsection{Basisvereisten}

\subsection{Extra functionaliteit}

\section{Planning}

\section{Appendix}

\end{document}