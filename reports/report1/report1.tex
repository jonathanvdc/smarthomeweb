\documentclass[12pt,a4paper]{article}
\usepackage[utf8]{inputenc}
\usepackage{amsmath}
\usepackage{amsfonts}
\usepackage{amssymb}
% \usepackage{all_is_fuck}
\begin{document}
\title{Rapport 1}
\author{Sibert Aerts \\ Ken Bauwens \\ Pieter Hendriks \\ Jonathan Van der Cruysse \\ Mauris Van Hauwe}
\maketitle

\newcommand{\code}[1]{\texttt{#1}}

\section{Status}

We hebben intussen de volgende features ge\"implementeerd:

\begin{itemize}

\item \textbf{REST API}: Jonathan, Mauris \& Sibert

\item \textbf{Database Structuur}: Jonathan, Pieter \& Mauris

\item \textbf{Database Queries}: Jonathan \& Pieter

\item \textbf{Database Integration}: Jonathan, Mauris, Pieter \& Sibert

\item \textbf{Login}: Mauris \& Pieter

\item \textbf{HTML Views}: Sibert, Ken \& Mauris

\end{itemize}

\section{Design}

\subsection{Keuzes}
% Korte motivatie van de belangrijkste keuzes in het ontwerp.

\subsection{Database schema}
% Een ER diagram van de databank, met alle entiteiten en relaties. Voorzie een 
% woordje uitleg bij de verschillende tabellen, relaties en extra constraints.
Het ER-diagram is toegevoegd als bijlage.

\subsection{UML schema}
% Een UML diagram van de source code, met alle classes/modules en dependencies. 
% Voorzie een woordje uitleg bij de verschillende classes en dependencies, 
% alsook welke libraries (of frameworks of tools) zijn gebruikt. Je kan een 
% class diagram maken, of een combinatie van bijv. een package diagram, en een 
% sequence diagram. Het UML schema zelf is toegevoegd als bijlage.

Het UML schema zelf is toegevoegd als bijlage. De \emph{back-end} bestaat uit de volgende componenten:

\begin{itemize}

\item \textbf{model classes}: het \emph{model} stelt tuples in de database voor. Concreet is het \emph{model} opgesplitst in twee soorten klassen.

\begin{itemize}

\item \textbf{identity classes}: stellen een tuple in de database voor met een unieke \emph{identifier}, die door de \emph{data connection} gegenereerd wordt. Hun data wordt voorgesteld door een instantie van een \emph{data class}. Zo zijn \code{Location}, \code{Person} en \code{Sensor} enkele voorbeelden van \emph{identity classes}.

\item \textbf{data classes}: stellen concrete data voor. Deze data komt niet noodzakelijk precies overeen met een tuple in de database, maar altijd wel met een bepaalde deelverzameling daarvan, zonder unieke identifier. Er bestaan twee voorname \emph{use cases} voor deze \emph{data objects}. Enerzijds worden ze gebruikt om, zij het in samenwerking met een \emph{identity object}, zij het onafhankelijk, informatie uit de database voor te stellen. Anderzijds kunnen ze ook gebruikt worden om data van objecten die nog toegevoegd moeten worden aan de database -- en dus nog geen unieke \emph{identifier} hebben -- voor te stellen. \code{PersonData}, \code{LocationData} en \code{Measurement} zijn voorbeelden van \emph{data classes}.

\end{itemize}

\item \textbf{data connection}: spant de brug tussen de database en de \emph{back-end}. De \code{DataConnection} klasse opent en sluit de database, en bedient zich van SQL queries om records aan de database toe te voegen, en deze later weer uit de database te halen. \code{DataConnection} vormt een laag abstractie bovenop directe communicatie met de database: database tuples worden meteen vertaald in \emph{model objects}, en omgekeerd.

\item \textbf{Nancy modules}: een \emph{module} voorziet \emph{routing}-functionaliteit; elke module geeft een lijst URL's op, en bepaalt welke \emph{content} de server doorstuurt of aanvaardt wanneer een HTTP \textsc{get}, \textsc{put} of \textsc{post} request voorkomt. Hier wordt geen onderscheid gemaakt tussen de API, dat door scripts een applicaties gebruikt kan worden, en de HTML pagina's, die in een browser bekeken kunnen worden. Elke \emph{module} heeft zijn eigen verantwoordelijkheid, en toegang (via de API) tot de verschillende \emph{tables} van de database wordt voorzien door verschillende modules. De \emph{modules} zelf komen echter nooit direct in contact met de database: ze manipuleren enkel \emph{model objects}, en laten het communiceren met de database over aan \code{DataConnection}.

\item \textbf{Configuratie en bootstrapping}: naast interactie met gebruikers en database, bestaan er ook nog enkele \emph{utility classes}, die de configuratie en het opstarten van de server organiseren. Daarnaast voorzien deze klassen ook extra functionaliteit in de vorm van \emph{user login}. 

\end{itemize}

\section{Product}

\subsection{Basisvereisten}
% Overzicht van de verschillende pagina’s, en achterliggende functionaliteit, 
% vanuit het enduser perspectief. Voorzie een woordje uitleg bij elke pagina, 
% en verwijs hierbij naar de basisvereisten uit de project opgave.

\subsection{Extra functionaliteit}
% Overzicht van de extra functionaliteit vanuit het end-user perspectief.

\section{Planning}
% Een overzicht van de planning van de belangrijkste functionaliteit, voor de 
% volgende presentatie.
De volgende grote stap is interactie: users kunnen momenteel inloggen, maar nog
geen data uploaden of communiceren met ij elke pagina, 
% en verwijs hierbij naar de basisvereisten uit de project opghun vrienden. Sommige van de tabellen 
die we in de database gedefinieerd hebben, worden nog niet gebruikt. Het design 
van de front-end is ook niet finaal.

\section{Appendix}
% Een listing van de SQL create statements, alsook niet-triviale SQL queries.
Het SQL-script met alle \textsc{create}-statements is toegevoegd als bijlage.

\end{document}