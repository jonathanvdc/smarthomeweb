\documentclass[12pt,a4paper]{article}
\usepackage[utf8]{inputenc}
\usepackage[dutch]{babel}
\usepackage{amsmath}
\usepackage{amsfonts}
\usepackage{amssymb}
\usepackage{listings}
\usepackage{DejaVuSansMono}
\usepackage{relsize}
% \usepackage{all_is_fuck}
\begin{document}
\title{Rapport 1}
\author{Sibert Aerts \\ Ken Bauwens \\ Pieter Hendriks \\ Jonathan Van der Cruysse \\ Mauris Van Hauwe}
\maketitle

\newcommand{\code}[1]{\texttt{#1}}
\newcommand{\CS}{{C\nolinebreak[4]\hspace{-.05em}\raisebox{.4ex}{\relsize{-2}{\textbf{\#}}}}}
\lstset{language=[Sharp]C,basicstyle=\ttfamily}


\section{Status}

We hebben intussen de volgende features ge\"implementeerd:

\begin{itemize}

\item \textbf{REST API}: Jonathan, Mauris \& Sibert

\item \textbf{Database Structuur}: Jonathan, Pieter \& Mauris

\item \textbf{Database Queries}: Jonathan \& Pieter

\item \textbf{Database Integration}: Jonathan, Mauris, Pieter \& Sibert

\item \textbf{Login}: Mauris \& Pieter

\item \textbf{HTML Views}: Sibert, Ken \& Mauris

\end{itemize}

\section{Design}

\subsection{Keuzes}
% Korte motivatie van de belangrijkste keuzes in het ontwerp.
We hebben gekozen voor de programmeertaal \CS{}. Deze taal is vergelijkbaar met
C++, maar is veel expressiever en meer memory-safe. \CS{} ondersteunt ook 
\textit{task-based parallellism}, wat ons toelaat heel eenvoudig requests 
asynchroon af te handelen.

Als framework voor onze web-applicatie gebruiken we \textit{Nancy}. Dit is een 
minimalistisch web-framework ge\"inspireerd door het Ruby-framework 
\textit{Sinatra}. De simpelste Nancy-applicatie ziet er als volgt uit:

\begin{lstlisting}
    public class MyModule : NancyModule {
        public MyModule() { Get["/"] = _ => "Hey!" }
    }
\end{lstlisting}


\subsection{Database schema}
% Een ER diagram van de databank, met alle entiteiten en relaties. Voorzie een 
% woordje uitleg bij de verschillende tabellen, relaties en extra constraints.
Het ER-diagram is toegevoegd als bijlage.

\subsection{UML schema}
% Een UML diagram van de source code, met alle classes/modules en dependencies. 
% Voorzie een woordje uitleg bij de verschillende classes en dependencies, 
% alsook welke libraries (of frameworks of tools) zijn gebruikt. Je kan een 
% class diagram maken, of een combinatie van bijv. een package diagram, en een 
% sequence diagram. Het UML schema zelf is toegevoegd als bijlage.

De backend bestaat uit de volgende componenten:

\begin{itemize}

\item \textbf{model classes}:  

\item \textbf{}:

\end{itemize}

\section{Product}

\subsection{Basisvereisten}
% Overzicht van de verschillende pagina’s, en achterliggende functionaliteit, 
% vanuit het enduser perspectief. Voorzie een woordje uitleg bij elke pagina, 
% en verwijs hierbij naar de basisvereisten uit de project opgave.

\subsection{Extra functionaliteit}
% Overzicht van de extra functionaliteit vanuit het end-user perspectief.

\section{Planning}
% Een overzicht van de planning van de belangrijkste functionaliteit, voor de 
% volgende presentatie.
De volgende grote stap is interactie: users kunnen momenteel inloggen, maar nog
geen data uploaden of communiceren met hun vrienden. Sommige van de tabellen 
die we in de database gedefinieerd hebben, worden nog niet gebruikt. Het design 
van de front-end is ook niet finaal.

\section{Appendix}
% Een listing van de SQL create statements, alsook niet-triviale SQL queries.
Het SQL-script met alle \textsc{create}-statements is toegevoegd als bijlage.

\end{document}