\documentclass[10pt,a4paper]{article}
\usepackage[latin1]{inputenc}
\usepackage[dutch]{babel}
\usepackage{amsmath}
\usepackage{amsfonts}
\usepackage{amssymb}
\usepackage{graphicx}
%\usepackage{pxfonts}
\usepackage{microtype}
\usepackage{xcolor}
\usepackage{enumerate}
\author{Sibert Aerts\\
Ken Bauwens\\
Pieter Hendriks\\
Jonathan Van der Cruysse\\
Mauris Van Hauwe}
\title{Project Databases: Verslag 4}
\date{}
\newcommand{\red}[1]{{\color{red!80!green}#1}}
\newcommand{\green}[1]{{\color{green!60!blue}#1}}

\begin{document}
  \maketitle
  \section{Aanwezigheden}
  Iedereen was aanwezig op de vergadering van 3 maart, op Sibert na, die ziek 
  was.
  
  \section{Status en onderwerpen}
  Deze week gaan we de API verder uit te breiden, om het aangepaste 
  \emph{database schema} beter te benutten. Daarnaast hebben we besloten om van 
  de \emph{front-end} een prioriteit te maken. 
  
  \section{Afspraken en planning}
  Puntjes voor volgende week:
  \begin{itemize}
    \item Ken en Sibert gaan views schrijven, en nadenken over hoe onze 
    \emph{front-end} eruit zal zien.
    \item Mauris voorziet een REST API voor de overige elementen in de 
    database.  
    \item Jonathan voorziet \emph{model classes} om de nieuwe en uitgebreide 
    database tables voor te stellen \emph{in memory}. Daarnaast poogt hij ook 
    eens een rudimentaire vorm van aggregatie te implementeren.
    \item Pieter gaat op ontdekkingsreis in de wondere wereld van de 
    \emph{authentication}.
  \end{itemize}
\end{document}
