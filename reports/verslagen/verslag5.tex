\documentclass[10pt,a4paper]{article}
\usepackage[latin1]{inputenc}
\usepackage[dutch]{babel}
\usepackage{amsmath}
\usepackage{amsfonts}
\usepackage{amssymb}
\usepackage{graphicx}
%\usepackage{pxfonts}
\usepackage{microtype}
\usepackage{xcolor}
\usepackage{enumerate}
\author{Sibert Aerts\\
Ken Bauwens\\
Pieter Hendriks\\
Jonathan Van der Cruysse\\
Mauris Van Hauwe}
\title{Project Databases: Verslag 5}
\date{}
\newcommand{\red}[1]{{\color{red!80!green}#1}}
\newcommand{\green}[1]{{\color{green!60!blue}#1}}

\begin{document}
  \maketitle
  \section{Aanwezigheden}
  Iedereen was aanwezig op de vergadering van 10 maart, op Sibert na.
  
  \section{Status en onderwerpen}
  Deze week is de \emph{authentication} boilerplate code geschreven. 
  Het grondwerk voor de \emph{front-end} is gelegd, we kunnen onze database records weergeven in de web interface.
  Er zijn nieuwe \emph{model} classes aangemaakt om de data op te slaan in memory.
  De \emph{REST API} werd verder uitgewerkt, om de database beter te reflecteren.
  
  \section{Afspraken en planning}
  Puntjes voor volgende week:
  \begin{itemize}
    
    \item Basis implementatie van grafieken voor de data representatie aan die client (Ken).
    
    \item Messages POSTen voor communicatie tussen de gebruikers (Sibert).
    
    \item Implementatie van measurement API en andere sensor-gerelateerde functionaliteit. (Jonathan).
    
    \item De CSV parser updaten, zodat deze aan de \emph{client-side} JSON aanmaakt in plaats van SQL, die dan via de interface gePOST kan worden (Pieter).
    
    \item De authentication implementatie verder zetten, zodat deze overeenkomt met users in de database (Mauris).
  \end{itemize}
\end{document}