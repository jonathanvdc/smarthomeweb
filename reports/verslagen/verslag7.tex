\documentclass[10pt,a4paper]{article}
\usepackage[latin1]{inputenc}
\usepackage[dutch]{babel}
\usepackage{amsmath}
\usepackage{amsfonts}
\usepackage{amssymb}
\usepackage{graphicx}
%\usepackage{pxfonts}
\usepackage{microtype}
\usepackage{xcolor}
\usepackage{enumerate}
\author{Sibert Aerts\\
Ken Bauwens\\
Pieter Hendriks\\
Jonathan Van der Cruysse\\
Mauris Van Hauwe}
\title{Project Databases: Verslag 7}
\date{}
\newcommand{\red}[1]{{\color{red!80!green}#1}}
\newcommand{\green}[1]{{\color{green!60!blue}#1}}

\begin{document}
  \maketitle
  \section{Aanwezigheden}
  Iedereen behalve Sibert was aanwezig op de vergadering van 14 april. 
  
  \section{Status en onderwerpen}
  Tijdens de paasvakantie hebben we aggregatie  van historische data ge\"implemen\-teerd. 
  
  De \emph{front-end} is ook uitgebreid met een aantal pagina's, en de \emph{root page} is wat opgesmukt. Daarnaast hebben we \emph{localization} toegevoegd: de site is nu tweetalig, Engels en Nederlands.
  
  Verder worden queries voor de database nu gegroepeerd in \emph{transactions}, waardoor grote \emph{inserts} veel effici\"enter behandeld kunnen worden.
  
  \section{Afspraken en planning}
  Voor volgende week staan deze punten op de agenda:
  
  \begin{itemize}
  	
  	\item Verder vertalen van de \emph{front-end}. (Mauris)
  	
  	\item Outliers elimineren uit de historische data, tijdens aggregatie. (Mauris)
  	
  	\item Database-functionaliteit en REST APIs voor: (Jonathan)
  	
  	\begin{itemize}
  		
  		\item \emph{Tags} voor sensoren.
  		
  		\item De kost per eenheid voor specifieke locaties.
  		
  		\item Het filteren van metingen en sensoren op \emph{tags}, verbruik en kostprijs.
  		
  		\item Het filteren van metingen die uit een bepaalde periode komen. 
  		
  	\end{itemize}
  	
  	\item Admin-mode aanmaken. (Sibert)
  	
  	\item Het uitbreiden, verfraaien en coherenter maken van de \emph{front-end}. (Sibert)
  	
  	\item Geaggregeerde data tonen in de \emph{front-end}. (Ken)
  	
  	\item Beschrijvingen aan sensoren toevoegen, en opmerkingen aan metingen. (Ken)
  	
  	\item Het sociale aspect van de site verbeteren: een ``wall'' maken, en daarop berichten tonen en delen. Daarnaast ook grafieken aan de ``wall'' toevoegen. (Pieter)
  	
  \end{itemize}
  
\end{document}