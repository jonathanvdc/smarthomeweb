\documentclass[10pt,a4paper]{article}
\usepackage[latin1]{inputenc}
\usepackage[dutch]{babel}
\usepackage{amsmath}
\usepackage{amsfonts}
\usepackage{amssymb}
\usepackage{graphicx}
%\usepackage{pxfonts}
\usepackage{microtype}
\usepackage{xcolor}
\usepackage{enumerate}
\author{Sibert Aerts\\
Ken Bauwens\\
Pieter Hendriks\\
Jonathan Van der Cruysse\\
Mauris Van Hauwe}
\title{Project Databases: Verslag 6}
\date{}
\newcommand{\red}[1]{{\color{red!80!green}#1}}
\newcommand{\green}[1]{{\color{green!60!blue}#1}}

\begin{document}
  \maketitle
  \section{Aanwezigheden}
  Iedereen was aanwezig op de vergadering van 24 maart. 
  
  \section{Status en onderwerpen}
  Deze week hebben we een logo voor de SmartHomeWeb website gemaakt. Daarnaast hebben we ook verder gewerkt aan het sociale aspect van de website. Zo kunnen gebruikers intussen messages versturen naar elkaar. Bovendien is het nu ook mogelijk om locaties met gebruikers te associ\"eren vanuit de \emph{front-end}.
  
  \section{Afspraken en planning}
  Dit is het laatste rapport v\'o\'or de paasvakantie. De taken hieronder slaan dus op de volgende twee weken.
  
  \begin{itemize}
  	\item Aggregatie van metingen over de loop van uren en dagen. (Mauris)
  	
  	\item De ``profile''-pagina voor de \emph{front-end}. (Sibert)
  	
  	\item Registreren van nieuwe gebruikers. (Jonathan)
  	
  	\item Database access groeperen in \emph{transactions}. (Jonathan)
  	
  	\item Het genereren van measurements beter integreren met de \emph{back-end}. (Ken)
  	
  \end{itemize}
  
  Pieter gaat op reis tijdens de paasvakantie, en werkt daarna verder aan het project.
  
\end{document}