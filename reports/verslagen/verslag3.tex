\documentclass[10pt,a4paper]{article}
\usepackage[latin1]{inputenc}
\usepackage[dutch]{babel}
\usepackage{amsmath}
\usepackage{amsfonts}
\usepackage{amssymb}
\usepackage{graphicx}
%\usepackage{pxfonts}
\usepackage{microtype}
\usepackage{xcolor}
\usepackage{enumerate}
\author{Sibert Aerts\\
Ken Bauwens\\
Pieter Hendriks\\
Jonathan Van der Cruysse\\
Mauris Van Hauwe}
\title{Project Databases: Verslag 3}
\date{}
\newcommand{\red}[1]{{\color{red!80!green}#1}}
\newcommand{\green}[1]{{\color{green!60!blue}#1}}

\begin{document}
  \maketitle
  \section{Aanwezigheden}
  Iedereen was aanwezig op de vergadering van 25 februari.
  
  \section{Status en onderwerpen}
  We hebben de ORM \emph{from scratch} herschreven, om rechtstreekser met de 
  database te communiceren. Onze werkende view hebben we gemigreerd naar het 
  \emph{Razor View Engine}: het eenvoudige engine dat we daarvoor gebruikten 
  was te gelimiteerd, en \emph{Razor} heeft betere localization support.
  
  De API werkt ook al een beetje: er is werkende JSON-serializatie, en er zijn 
  GET en POST requests voor gebruikers en locaties, die we eenvoudig zullen 
  kunnen uitbreiden naar de rest van de database. Nu moeten we wat meer focus 
  leggen op de \emph{front-end}.
  
  \section{Afspraken en planning}
  Puntjes voor volgende week:
  \begin{itemize}
    \item De \emph{front-end} uitbreiden met het nieuwe view engine (Ken, 
    Sibert).
    \item Nuttige functies schrijven die database-queries voorstellen 
    (Jonathan).
    \item De API uitbreiden (Mauris).
    \item Database-tabellen maken voor metingen en locaties (Pieter).
    \item Algemene projectstructuur verbeteren.
  \end{itemize}
\end{document}