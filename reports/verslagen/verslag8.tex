\documentclass[10pt,a4paper]{article}
\usepackage[latin1]{inputenc}
\usepackage[dutch]{babel}
\usepackage{amsmath}
\usepackage{amsfonts}
\usepackage{amssymb}
\usepackage{graphicx}
%\usepackage{pxfonts}
\usepackage{microtype}
\usepackage{xcolor}
\usepackage{enumerate}
\author{Sibert Aerts\\
Ken Bauwens\\
Pieter Hendriks\\
Jonathan Van der Cruysse\\
Mauris Van Hauwe}
\title{Project Databases: Verslag 8}
\date{}
\newcommand{\red}[1]{{\color{red!80!green}#1}}
\newcommand{\green}[1]{{\color{green!60!blue}#1}}

\begin{document}
  \maketitle
  \section{Aanwezigheden}
  Iedereen was aanwezig op de vergadering van 21 april. 
  
  \section{Status en onderwerpen}
  
  Sinds deze week kunnen we tags toevoegen aan, en wegnemen van, sensoren. Dit is intussen zowel in de \emph{front-end} als in de \emph{back-end} ge\"implementeerd.
  
  Daarnaast kunnen we nu ook data aanpassen van personen en sensoren. De \emph{front-end} is inmiddels zo goed als volledig tweetalig.
  
  Verder hebben we ook gewerkt aan aggregatie: die gebeurt nu ook per maand en per jaar. Aggregatie gebeurt momenteel nog volledig \emph{on-demand}, om ook met de meest recente data rekening te kunnen houden. Bij een \textsc{insert} in de database wordt de aggregatie-cache voor het uur, dag, maand en jaar van de measurement ge\"invalideerd.
  
  Outliers worden nu conform met de specificatie uit de metingen verwijderd.
  
  \section{Afspraken en planning}
  Voor volgende week staan deze punten op de agenda:
  
  \begin{itemize}
  	
  	\item Sensoren filteren op \emph{tags}. (Sibert)
  	
  	\item Data van meerdere sensoren tegelijkertijd beschouwen. (Ken)
  	
  	\item Samenvoegen van verbruik van alle gebruikers, voor admins. (Jonathan)
  	
  	\item Coherenter maken van de \emph{front-end}. (Sibert, Pieter en Jonathan)
  	
  	\item Verderzetting van taak vorige week: admin-mode aanmaken. (Mauris en Sibert)
  	
  	\item Verderzetting van taak vorige week: het sociale aspect van de site verbeteren: een ``wall'' maken, en daarop berichten tonen en delen. Daarnaast ook grafieken aan de ``wall'' toevoegen. (Pieter)
  	
  \end{itemize}
  
\end{document}