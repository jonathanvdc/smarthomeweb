\documentclass[10pt,a4paper]{article}
\usepackage[latin1]{inputenc}
\usepackage[dutch]{babel}
\usepackage{amsmath}
\usepackage{amsfonts}
\usepackage{amssymb}
\usepackage{graphicx}
%\usepackage{pxfonts}
\usepackage{microtype}
\usepackage{xcolor}
\usepackage{enumerate}
\author{Sibert Aerts\\
Ken Bauwens\\
Pieter Hendriks\\
Jonathan Van der Cruysse\\
Mauris Van Hauwe}
\title{Project Databases: Verslag 1}
\newcommand{\red}[1]{{\color{red!80!green}#1}}
\newcommand{\green}[1]{{\color{green!60!blue}#1}}

\begin{document}
  \maketitle
  \section{Aanwezigheden}
  Iedereen was aanwezig op de vergadering van 18 februari, behalve Pieter, die 
  afwezig was wegens ziekte.
  
  \section{Status en onderwerpen}
  We hebben heel eenvoudige werkende views, en kunnen vanuit de webserver 
  communiceren met de database.
  
  \section{Afspraken en planning}
  Puntjes voor volgende week:
  \begin{itemize}
    \item Momenteel biedt onze server enkel een eenvoudige \texttt{<h1>Hello 
    world!</h1>} pagina aan. Sibert en Pieter gaan kijken hoe we complexere 
    views kunnen implementeren: het Nancy framework biedt eenvoudige 
    \textsc{html}-templating aan 
    via het \textit{Super Simple View Engine}, en waarschijnlijk gaan we dat 
    gebruiken.
    \item Onze server moet reageren op meer dan alleen 
    \textsc{get}-requests. Jonathan gaat wat we nu hebben uitbreiden zodat 
    \textsc{put} ook werkt; dan staan we al dichter bij werkende \textsc{crud}.
    \item De basis van de database en de server werkt nu, dus we moeten ze met 
    elkaar laten communiceren. Hier zorgt Mauris voor.
    \item Sibert en Ken gaan hun C\#-kennis wat opfrissen, zodat iedereen vlot 
    kan samenwerken aan het project.
  \end{itemize}
\end{document}