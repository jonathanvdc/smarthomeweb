\documentclass[10pt,a4paper]{article}
\usepackage[latin1]{inputenc}
\usepackage[dutch]{babel}
\usepackage{amsmath}
\usepackage{amsfonts}
\usepackage{amssymb}
\usepackage{graphicx}
%\usepackage{pxfonts}
\usepackage{microtype}
\usepackage{xcolor}
\usepackage{enumerate}
\author{Sibert Aerts\\
Ken Bauwens\\
Pieter Hendriks\\
Jonathan Van der Cruysse\\
Mauris Van Hauwe}
\title{Project Databases: Verslag 2}
\newcommand{\red}[1]{{\color{red!80!green}#1}}
\newcommand{\green}[1]{{\color{green!60!blue}#1}}

\begin{document}
  \maketitle
  \section{Aanwezigheden}
  Iedereen was aanwezig op de vergadering van 18 februari, behalve Pieter, die 
  afwezig was wegens ziekte.
  
  \section{Status en onderwerpen}
  We hebben een heel eenvoudige (werkende) view, en kunnen vanuit de webserver
  communiceren met de database: queries werken, en we hebben een eenvoudige
  \textsc{orm} waarmee we users uit de database mappen naar onze eigen klasse.
  
  \section{Afspraken en planning}
  Puntjes voor volgende week:
  \begin{itemize}
    \item De ORM uitbreiden: welke tabellen hebben we 
    nog nodig, en hoe stellen we de rijen ervan voor als C\# objecten? Jonathan 
    zorgt hiervoor.
    \item Nu we kunnen communiceren met SQLite, is het tijd om te beginnen aan 
    een REST API. Mauris zorgt hiervoor.
    \item Ken en Sibert gaan views schrijven, en nadenken over hoe onze 
    frontend eruit zal zien.
    \item Pieter gaat de database uitbreiden op basis van de ElecSim-data, 
    zodat ons model de output ervan kan verwerken.
  \end{itemize}
\end{document}