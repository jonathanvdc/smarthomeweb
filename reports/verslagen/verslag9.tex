\documentclass[10pt,a4paper]{article}
\usepackage[latin1]{inputenc}
\usepackage[dutch]{babel}
\usepackage{amsmath}
\usepackage{amsfonts}
\usepackage{amssymb}
\usepackage{graphicx}
\usepackage{pxfonts}
\usepackage{microtype}
\usepackage{xcolor}
\usepackage{enumerate}
\author{Sibert Aerts\\
Ken Bauwens\\
Pieter Hendriks\\
Jonathan Van der Cruysse\\
Mauris Van Hauwe}
\title{Project Databases: Verslag 9}
\date{}
\newcommand{\red}[1]{{\color{red!80!green}#1}}
\newcommand{\green}[1]{{\color{green!60!blue}#1}}

\begin{document}
  \maketitle
  \section{Aanwezigheden}
  Iedereen was aanwezig op de vergadering van 12 mei. 
  
  \section{Status en onderwerpen}
  
  Vorige weken zijn de volgende zaken gebeurd:
  
  \begin{itemize}
  	\item Ken heeft $k$-means clustering ge\"implementeerd.
  
	\item Jonathan is begonnen aan het refactoren van grafieken: het \emph{client-side} \emph{model} en de \emph{view} zijn nu duidelijk van elkaar gescheiden.
	
	\item De encoding van \emph{saved graphs} in de database is nu een pak effici\"enter.
	
	\item Mauris heeft een kleine schoonmaak van de repository gedaan: oude code is verwijderd, en ongebruikte files ook.
  
  \end{itemize}
  
  \section{Afspraken en planning}
  Voor volgende week staan deze punten op de agenda:
  
  \begin{itemize}
  	
  	\item $k$-means clustering afwerken. (Ken)
  	
  	\item Predictie van totaalverbruik. (Mauris)
  	
  	\item Een aparte \emph{compare}-pagina maken voor grafieken. (Sibert)
  	
  	\item Het herwerken van de \emph{dashboard} om het nieuwe grafiek-\emph{model} te gebruiken, enkel voor \emph{grouping}. (Jonathan)
  	
  	\item Het updaten van de \emph{wall} UI om \emph{interactive graphs} te tonen. (Pieter)
  	
  	\item Het toevoegen van een timer aan de \emph{graph control}, zodat nieuwe \emph{measurements} \emph{live} getoond kunnen worden, zonder dat een \emph{refresh} nodig is. (Mauris)
  	
  \end{itemize}
  
\end{document}